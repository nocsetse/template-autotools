% !TeX root = paper-autotools-python.tex
% !TeX TXS-program:compile = txs:///pdflatex/[--shell-escape]
% https://orcid.org/0000-0003-4586-8500

\include{preamble}

\begin{document}

\title{\mytitle}
\author[1,2]{Franklin E. Diaz\\ }
%\texttt\href{emailto: fdiaz@paloaltonetworks.com}{fdiaz@paloaltonetworks.com}}
%\affil[1]{Palo Alto Networks}
%\affil[2]{Professional Services - Automation}
\begin{titlepage}
	\date{June 03, 2022}
	\maketitle
	%\thispagestyle{empty} % remove page number from title page?
	\begin{abstract}
		\raggedright
		The goal of this document is to detail my use of GNU Autotools with Python development.\\
		\vspace{4cm}
		\centering{\href{https://github.com/devsecfranklin/paper-autotools-python/blob/main/paper-autotools-python.pdf}{ Click here to download latest version.}}

	\end{abstract}
\end{titlepage}

\begin{comment}
Source files for this paper are available: \url{https://github.com/devsecfranklin/paper-autotools-python/}
\end{comment}


\section{\label{sec:Start}Using GNU Autotools with Python Projects}
\vspace{2mm}
\justifying


\begin{figure}[ht]
    \includegraphics[width=12cm]{images/diagram.png}
    \caption{A basic overview of how the main autotools components fit together.}
    \label{diagram}
\end{figure}

Source: \href{https://devmanual.gentoo.org/general-concepts/autotools/index.html}{Gentoo Linux DevManual: The Basics of Autotools}

\section{\label{sec:bootstrap}The bootstrap.sh script}

The purpose of the ``bootstrap.sh'' script is to allow project maintainers to prepare the local environment for use of autotools.
The boostrap script will attempt to guess if the local host is running a certain Linux distribution, or MacOS. There are execution
paths for each of these respective scenarios.

At first execution, autotools will run libtool, automake, and configure. Autotools creates a file named ``config.log''. If the config.log
file is found on subsequent runs of the script, fewer configuration steps will be performed since a certain system state is assumed. To
get a ``clean start'' the maintainer can simply delete the config.log file and rerun the bootstrap.sh script.

\justifying
A full working example of the bootstrap.sh script can be found in the source repository for this paper.

\section{\label{sec:config}The configure.ac file}
\vspace{2mm}
\justifying
A ``./configure'' script can be generated from a template named ``configure.ac''. This template is comprised of macros that are used to identify local system
software and program locations. In our case, we are interested in identifying the location of the Python 3.x installation on the local system.

\justifying
\href{https://www.gnu.org/software/automake/manual/html\_node/Python.html}{Per the automake documentation} we can define the minimum acceptable version of Python
and a variable that automake can use to refer to the version of Python discovered on the system. These declarative statements and the resultant values can then be
referenced by automake and added explicitly to the ``Makefile.am'' template.

\begin{mybox}{\thetcbcounter: Declaring Python Autoconf macros in configure.ac}
    \lstinputlisting{code/configure.ac.1}
\end{mybox}
\vspace{2mm}

\justifying
A full working example of the configure.ac file can be found in the source repository for this paper.

\section{\label{sec:make}The Makefile.am file}


\end{document}
